\chapter{Introducción}

El mundo digital dominado por datos requiere la utilización de algoritmos, tecnologías y mecanismos que permitan orquestar procesamientos de grandes volúmenes de datos, los cuales pueden tardar horas e, incluso, días en completarse. Aún así, el diseño de la aplicación, junto con el flujo de trabajo requiere un esfuerzo considerable cuando se realizan estas aplicaciones.

Ahora bien, sería deseable desarrollar una plataforma que permita distribuir estos flujos de trabajo computacionalmente intensivos en un sistema distribuído con el fin de disminuir el tiempo de ejecución del flujo. También, es deseable disminuir el presupuesto utilizado, ya que es muy común que en los modelos de cómputo en la nube se tenga un modelo económico en que se cobre por utilizar estos servicios.

En este trabajo se muestra el desarrollo de esta plataforma de ejecución de flujos de trabajo intensivo en datos con aplicación en cómputo en la nube. También se muestra el marco teórico necesario para el desarrollo de este proyecto.

Primero, empezaremos por describir el modelo de costos de cómputo en la nube, marcado por acuerdos de nivel de servicio, y se describirá un mecanismo para poder

Luego, describiremos cómo hacer benchmarking dentro de plataformas de cómputo en la nube. Se sugiere utilizar el benchmark proporcionado por LINPACK, ya que éste puede aproximar el rendimiento de un sólo núcleo y varios núcleos. En la tesis de XXX et al. se puede mostrar. Luego describiremos el algoritmo de planificación de flujos de trabajo. Luego describiremos el mecanismo para poner las tareas en de. Luego, compararemos con soluciones existentes como Aneka y HTCondor


\section{Descripción del problema}

Se tiene un flujo de trabajo definido por $W = (T, D)$, donde $T$ es el conjunto de tareas del flujo de trabajo y $D$ es el conjunto de dependencias que existen entre las tareas del flujo de trabajo. Se desea planificar las tareas en un conjunto de recursos $R$ en la nube de diferente capacidad, los cuales están conectados entre sí por medio de una red privada virtual. Se tiene una función $F: W \times R \mapsto (\mathcal{R}^{+} \times {R}^{+}) $ la cual asocia un costo y un tiempo total de ejecución a una planificación de este flujo de trabajo con el conjunto de recursos dato. Se requiere minimizar la función $F$ tanto en costo como en tiempo.
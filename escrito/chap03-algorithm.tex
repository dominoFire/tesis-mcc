\chapter{Algoritmo de planificación de flujos de trabajo sobre cómputo en la nube}

En este capítulo, describiremos el algoritmo que es la pieza clave del sistema para agendar tareas de un flujo de trabajo en infraestructura de cómputo en la nube, tomando en cuenta los lineamientos de diseño propuestos anteriormente.

También, un área de investigación activa es la planificación de tareas a nivel de procesador (CPU's), en donde se asume que se tiene un procesador cuyos núcleos son idénticos e intercambiables. Además, en los trabajos de investigación referentes a esta área se consideran a los grafos dirigidos acíclicos como los modelos de tareas más generales. Así, basados en el trabajo de Saifullah et al. \cite{saifullah2013multi}, en donde primero se transforma un grafo dirigido acíclico en un conjunto de tareas secuenciales con secciones que se ejecutan en paralelo.

Primero, es muy común que los servicios de cómputo en la nube se paguen la utilización de las máquinas virtuales por hora, dejando sin cobrar porciones de hora no utlizadas. Además, cada recurso es n

% Podemos hacer esto aún más abstracto y combinarlo con la teoría de Pinedo

% EFT: Earliest-Finish Time
% DM: Deadline Monotonic
% EDF: Earliest-Deadline First
% Global EDF
% Partitioned EDF

% how to enable theory with processor teory
% I think my thesis is that weird link, with the basic paper
% no hay que olvidar que hay que optimizar tiempo, dinero

% ya sabemos que es NP-hard, asi que nos enfocaremos en el problema

% Primero, tienes recursos 'infinitos', solo sujetos a presupuesto
% puedes averiguar el numero maximo de maquinas a levantar con el modelo de tareas paralelar
% simplemente cuentas el segmento con el mayor numero de paralelizaciones


% De acuerdo al primer capitulo del Pinedo, si hay esfuerzos para clasificar
% los problemas

% Hacer puente entre las teorías de Buyya y Jia Yu y las teorías de Pinedo


% Hay que leer los papers de arquitectura

\chapter{Algoritmo de planificación de flujos de trabajo sobre cómputo en la nube}

En este capítulo, describiremos el algoritmo que es la pieza clave del sistema para agendar tareas de un flujo de trabajo en infraestructura de cómputo en la nube, tomando en cuenta los lineamientos de diseño propuestos anteriormente.

Primero, es muy común que los servicios de cómputo en la nube se paguen la utilización de las máquinas virtuales por hora, dejando sin cobrar porciones de hora no utlizadas.

También, un área de investigación activa es la planificación de tareas a nivel de procesador (CPU's), en donde se asume que se tiene un procesador cuyos núcleos son idénticos e intercambiables. Además, en los trabajos de investigación referentes a esta área se consideran a los grafos dirigidos acíclicos como los modelos de tareas más generales. Así, basados en el trabajo de Saifullah et al. \cite{saifullah2013multi}, en donde primero se transforma un grafo dirigido acíclico en un conjunto de tareas secuenciales con secciones que se ejecutan en paralelo.

% how to enable theory with processor teory
% I think my thesis is that weird link, with the basic paper 

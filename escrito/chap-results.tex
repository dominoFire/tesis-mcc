\chapter{Resultados}

En este capitulo presentamos algunos resultados del algoritmo ciego y algunas comparaciones con otros algoritmos de planificación. 

Para probar el algoritmo ciego, se generaron aleatoriamente flujos de trabajo utilizando el simulador de flujos de trabajo. El objetivo de estas simulaciones es comparar cuán óptimas son las planificaciones que genera el algoritmo de planificación propuesto en este trabajo respecto a los algoritmos MaxMin, MinMin y Miope. Cabe aclarar que para estas pruebas, el número y el tipo de recursos es determinado por el algoritmo ciego. Luego, se generan otras planificaciones con los algoritmos MaxMin, MinMin y Miope. También, es importante notar que a todos los algoritmos se les pidió encontrar la planificación con el menor tiempo total de ejecución (makespan).


Para las pruebas, se generaron 50 flujos de trabajo, con 10 tareas. Los factores de complejidad varían entre 50 y 100. Cada flujo de trabajo fue generado con base en un generador de números aleatorios congruencial, utilizando una semilla diferente para cada grafo. Dado que existe una componente aleatoria en la generación, se hicieron varias pruebas.

Después de correr 30 pruebas, en el 10\% de los resultados el algoritmo ciego encontraba planificaciones con menor tiempo que alguno de los tres algoritmos comparados. Ver tabla de anexos.

\section{Discusión de resultados}

Esto se debe a que hay algunos detalles con el algoritmo

1) Está pensado para ser más flexible y/o extensible.
2)

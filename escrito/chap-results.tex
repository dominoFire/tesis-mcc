\chapter{Resultados}

En este capitulo presentamos algunos resultados del algoritmo ciego y algunas comparaciones con otros algoritmos de planificación. 

Para probar el algoritmo ciego, se generaron aleatoriamente flujos de trabajo utilizando el simulador de flujos de trabajo. El objetivo de estas simulaciones es comparar cuán óptimas son las planificaciones que genera el algoritmo de planificación propuesto en este trabajo respecto a los algoritmos MaxMin, MinMin y Miope. Cabe aclarar que para estas pruebas, el número y el tipo de recursos es determinado por el algoritmo ciego. Luego, utilizando estos recursos sugeridos por el algoritmo ciego, se generan otras planificaciones con los algoritmos MaxMin, MinMin y Miope. También, es importante notar que a todos los algoritmos se les pidió encontrar la planificación con el menor tiempo total de ejecución (makespan). 

Para las pruebas, se generaron 100 flujos de trabajo con un número variable de tareas, que van desde 1 hasta 50 tareas. Los factores de complejidad de las tareas varían entre 50 y 100. Cada flujo de trabajo fue generado con base en un generador de números aleatorios congruencial, utilizando una semilla diferente para cada flujo de trabajo.

Para evaluar el desempeño de los algoritmos de planificación, se tomaron las planificaciones que arrojaba cada algortmo y con ellas, se calculó el makespan y el costo total de ejecución, definido como el producto del costo por hora de mantener una máquina virtual por el periodo de tiempo por el que la máguina se encuentra prendida, es decir, la diferencia entre el tiempo final de la última tarea a ejecutar por la máquina y el tiempo inicial de la primera tarea a ejecutar.

En tablas \ref{table:makespans} y \ref{table:costs}, se pueden ver los resultados de las ejecuciones. Como se puede apreciar en las tablas, en los flujos de trabajo número 63, 93 y 95, el algoritmo ciego obtuvo mejores costos de ejecución en comparación con los algoritmos MaxMin, MinMin y Miope. Sin embargo, el algoritmo ciego no pudo obtener una mejoría absoluta en cuanto a tiempo total de ejecución con respecto a los demás algoritmos de planificación. 

Por otra parte, el algoritmo ciego puede ser modificado para utilizar cualquier función de costo parcial que se quiera minimizar. En el caso de esta implementación, se implementaron dos funciones de costo parcial: tiempo total de ejecución y costo total de ejecución. De esta forma, se generaron 100 flujos de trabajo aleatorios y se aplicaron las mismas pruebas que el proceso anterior, variando la funcion de costo parcial para minimizar el costo de ejecución. Para comparar resultados con los demás algoritmos, se programaron que obtuvieran la planificación con el menor tiempo total de ejecución. Si bien, esta prueba no nos puede decir qué tan óptimas son las planificaciones del algoritmo ciego respecto a las planificaciones de los algoritmos MaxMin, MinMin y Miope, esta prueba nos da un indicio sobre qué tan óptimas son las planificaciones del algoritmo ciego en cuanto a tiempo de ejecución. En las tablas XXX y XXX podemos ver dicha métrica. 




\section{Discusión de resultados}

Esto se debe a que hay algunos detalles con el algoritmo

1) Está pensado para ser más flexible y/o extensible.
2)

\chapter{Resultados}

En este capitulo presentamos algunos resultados del algoritmo ciego y algunas comparaciones con otros algoritmos de planificación. 

Para probar el algoritmo ciego, se generaron aleatoriamente flujos de trabajo utilizando el simulador de flujos de trabajo. El objetivo de estas simulaciones es comparar cuán óptimas son las planificaciones que genera el algoritmo de planificación propuesto en este trabajo respecto a los algoritmos MaxMin, MinMin y Miope. Cabe aclarar que para estas pruebas, el número y el tipo de recursos es determinado por el algoritmo ciego. Luego, utilizando estos recursos sugeridos por el algoritmo ciego, se generan otras planificaciones con los algoritmos MaxMin, MinMin y Miope. También, es importante notar que a todos los algoritmos se les pidió encontrar la planificación con el menor tiempo total de ejecución (makespan). 

Para las pruebas, se generaron 50 flujos de trabajo con 10 tareas cada uno. Los factores de complejidad de las tareas varían entre 50 y 100. Cada flujo de trabajo fue generado con base en un generador de números aleatorios congruencial, utilizando una semilla diferente para cada flujo de trabajo.

Para evaluar el desempeño de los algoritmos de planificación, se tomaron las planificaciones que arrojaba cada algortmo y con ellas, se calculó el makespan y el costo total de ejecución, definido como el producto del costo por hora de mantener una máquina virtual por el periodo de tiempo por el que la máguina se encuentra prendida, es decir, la diferencia entre el tiempo final de la última tarea a ejecutar por la máquina y el tiempo inicial de la primera tarea a ejecutar.





\section{Discusión de resultados}

Esto se debe a que hay algunos detalles con el algoritmo

1) Está pensado para ser más flexible y/o extensible.
2)

\chapter{Conclusiones}
\label{chap:conclusions}

En este trabajo, se propuso un nuevo algoritmo de planificación de flujos de trabajo, tomando en cuenta su ejecución y despliegue en entornos de cómputo en la nube. Este algoritmo, a diferencia de los utilizados para comparar los resultados obtenidos, puede calcular un número mínimo de máquinas o recursos necesarios para correr el flujo de trabajo sin que alguna tarea del flujo de trabajo tenga que esperar por recursos. Otra ventaja de este algoritmo es que se puede utilizar una funci\'on de costo parcial, ya sea para optimizar costo, tiempo de ejecuci\'on u otros requisitos espec\'ificos.

Aunque el algoritmo ciego no genera optimizaciones mucho más eficientes que los algoritmos Miope, MaxMin y MinMin, el algoritmo propuesto en este trabajo da posibilidad a estimar el número de recursos necesarios para ejecutar el flujo de trabajo con máximo paralelismo; los algoritmos utilizados como comparación no realizan la estimación de recursos, solamente minimizan algún costo. Una posible causa de que el algoritmo ciego no tenga mejores resultados que los algoritmos comparados es que no se realiza ninguna optimización a la hora de unir las asignaciones de cada uno de los segmentos. Esto causaría que tareas que son secuenciales puedan ser asignadas a diferentes recursos, incurriendo en un costo adicional por no asignarlos a un solo recurso.

Otra ventaja del algoritmo ciego es su modularidad: es posible cambiar el procedimiento de asignación de tareas a configuraciones de recursos sin afectar la segmentación del flujo de trabajo; también se puede cambiar el procedimiento para unir las asignaciones de todos los segmentos en la planificación final. Finalmente, cabe aclarar que hay algunos flujos de trabajo en donde el algoritmo ciego obtiene mejores resultados de costo de ejecución y tiempo total de ejecución mejores que los resultados obtenidos a través de los algoritmos comparados.

Tambi\'en, en este trabajo se cre\'o un prototipo de un sistema administrador de flujos de trabajo en cómputo en la nube, llamado sweeper, con el objetivo de vislumbrar cu\'ales son las complejidades de construir un sistema de este tipo. Además, sweeper sirvió como una plataforma de prueba a una parte del algoritmo ciego. Sin duda, existen muchos retos t\'ecnicos a solventar, como la administraci\'on de las m\'aquinas virtuales, el almacenamiento y la distribuci\'on de resultados, la configuraci\'on de los programas a ejecutar, entre otros.


\section{Trabajo futuro}

Existen múltiples áreas de oportunidad para mejorar este algoritmo, entre ellas se encuentran:

\begin{enumerate}
\item{Una demostraci\'on formal de que el algoritmo ciego calcula el numero \'optimo de m\'aquinas necesarias para alcanzar el paralelismo no bloqueante.}
\item{Nuevas funciones de costo parcial para evaluar el comportamiento del algoritmo utilizando otras fuentes de optimizaci\'on.}
\end{enumerate}

Por otro lado, el prototipo tiene posibles l\'ineas de trabajo futuro, las cuales se mencionan a continuaci\'on:

\begin{enumerate}
\item{Pooling de m\'aquinas virtuales, debido a que es muy costoso (en t\'erminos de tiempo) prender y apagar m\'aquinas virtuales.}
\item{Entrega progresiva de resultados. El prototipo actual descarga los resultados de la ejecuci\'on una vez que termin\'o el flujo de trabajo; ser\'ia mejor que entregue resultados conforme se vayan obteniendo.}
\item{Perfiladores de recursos y de tareas integrados a sweeper. Los algoritmos de planificación de flujos de trabajo generan mejores asignaciones si conocen con certeza el tiempo de ejecución de las tareas en los recursos disponibles. Si el sistema administrador de flujos de trabajo puede obtener esta información a través del análisis del uso de recursos, entonces podrá hacer mejores asignaciones.}
\item{Utilizar contenedores. Recientemente, la virtualizaci\'on a nivel sistema operativo se ha vuelto popular para desplegar servicios en la nube. Su principal caracter\'istica es que pueden crearse m\'as r\'apido que una m\'aquina virtual y que pueden funcionar con un \emph{kernel} compartido.}
\end{enumerate}

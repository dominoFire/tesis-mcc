\begin{center}
Resumen
\end{center}
\noindent En este trabajo se desarroll\'o un algoritmo de planificaci\'on de flujos de trabajo con enfoque a entornos de c\'omputo en la nube, con la capacidad de calcular el número de máquinas necesarias para reducir el tiempo de espera de tareas por recursos disponibles. Tambi\'en, se implement\'o este algoritmo en un software simulador de flujos de trabajo. Los resultados experimentales muestran que el algoritmo reduce los costos de ejecuci\'on de mejor manera que los algoritmo Miope. Por \'ultimo, se implement\'o un sistema administrador de flujos de trabajo en la nube, llamado sweeper, con el objetivo de explorar cu\'ales son los retos a los que hay que enfrentarse para llevar este tipo de algoritmos de planifcaci\'on a entornos reales.
\\\\
\noindent \emph{Palabras clave}: Planificación de flujos de trabajo, Segmentación, Búsqueda exhaustiva, Optimizaci\'on de costos

\begin{center}
Abstract
\end{center}
\noindent In this work we developed a workflow scheduling algorithm with a focus on cloud computing environments. This novel algorithm can compute the number of machines required to reduce task waiting time for available resources. Also, this algorithm was implemented in a workflow simulator software. The experimental results show that the algorithm reduces execution costs better than the Myopic algorithm. Finally, a prototype cloud workflow manager system, called sweeper, was developed, with the goal of exploring what are the challenges we need to face in order to use those workflow scheduling algorithms on real life scenarios.
\\\\
\noindent \emph{Keywords}: Workflow scheduling, Segmentation, Exhaustive search, Cost optimizacion

\clearpage